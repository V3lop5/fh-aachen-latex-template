\chapter{Examples}
\label{chap:examples}

Notes are helpful! \todo{Finish this and that!} Simple cite example \cite{github} or a cite with some extra info \cite[S. 1]{github}.

\section{Image Examples}
To use the \kotlin{\\image} command please put the image directly in \kotlin{attachments/images/}. 

The README file (\imageref{example/README.png}) contains many helpful things. The Images are printed with 90 percent width.

\image{example/README.png}{Screenshot of README file}
\FloatBarrier 

For Large Images you can use \kotlin{\\imagebig} command.
\imagebig{example/Code.png}{Screenshot of code section}
\FloatBarrier 

If you want an extra description you can use \kotlin{\\imagedesc}.
\imagedesc{example/Release.png}{Screenshot of Releases}{This is my extra description \cite{github}}
\FloatBarrier 

By default images are placed somewhere. To force a Position, simply use \kotlin{\\FloatBarrier} after the \kotlin{\\image} command.


\todo{Insert examples}

\section{Code Examples}
This Section contains Code Examples. It uses the Minted Package \cite{minted}.

\subsection{Code from File}
You can insert code from a file. The following code was inserted by File:
\kotlinfile{HelloWorld.kt}{Code from file}

\subsection{Inline code}
The following word is inline code: \kotlin{fun getName(): String}

\section{Text File Examples}
\todo{Insert examples}